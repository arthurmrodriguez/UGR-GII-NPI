%%%%%%%%%%%%%%%%%%%%%%%%%%%%%%%%%%%%%%%%%
% Short Sectioned Assignment LaTeX Template Version 1.0 (5/5/12)
% This template has been downloaded from: http://www.LaTeXTemplates.com
% Original author:  Frits Wenneker (http://www.howtotex.com)
% License: CC BY-NC-SA 3.0 (http://creativecommons.org/licenses/by-nc-sa/3.0/)
%%%%%%%%%%%%%%%%%%%%%%%%%%%%%%%%%%%%%%%%%

%----------------------------------------------------------------------------------------
%	PACKAGES AND OTHER DOCUMENT CONFIGURATIONS
%----------------------------------------------------------------------------------------

\documentclass[paper=a4, fontsize=11pt]{scrartcl} % A4 paper and 11pt font size

% ---- Entrada y salida de texto -----

\usepackage[T1]{fontenc} % Use 8-bit encoding that has 256 glyphs
\usepackage[utf8]{inputenc}
%\usepackage{fourier} % Use the Adobe Utopia font for the document - comment this line to return to the LaTeX default

\usepackage{xcolor}
\usepackage{fancyhdr}
\usepackage{listings, lstautogobble}
\usepackage{comment}

\usepackage{subfigure} % subfiguras

%----------
\lstdefinestyle{customc}{
	belowcaptionskip=1\baselineskip,
	autogobble=true,
	breaklines=true,
	frame=L,
	xleftmargin=\parindent,
	language=C,
	showstringspaces=false,
	basicstyle=\footnotesize\ttfamily,
	keywordstyle=\bfseries\color{green!40!black},
	commentstyle=\itshape\color{purple!40!black},
	identifierstyle=\color{blue},
	stringstyle=\color{orange},
}

\lstdefinestyle{bashmc}{
	frame=l,
	tabsize=2,
	breaklines=true,
	xleftmargin=\parindent,
	basicstyle=\ttfamily,
	showstringspaces=false,
	commentstyle=\color{red},
	keywordstyle=\color{blue},
	autogobble=true,
}





% ---- Idioma --------

\usepackage[spanish, es-tabla]{babel} % Selecciona el español para palabras introducidas automáticamente, p.ej. "septiembre" en la fecha y especifica que se use la palabra Tabla en vez de Cuadro

% ---- Otros paquetes ----
\usepackage{eurosym} % para el euro
\usepackage{url} % ,href} %para incluir URLs e hipervínculos dentro del texto (aunque hay que instalar href)
\usepackage{amsmath,amsfonts,amsthm} % Math packages
\usepackage{centernot}
%\usepackage{graphics,graphicx, floatrow} %para incluir imágenes y notas en las imágenes
\usepackage{graphics,graphicx, float} %para incluir imágenes y colocarlas

% Para hacer tablas comlejas
%\usepackage{multirow}
%\usepackage{threeparttable}

%\usepackage{sectsty} % Allows customizing section commands
%\allsectionsfont{\centering \normalfont\scshape} % Make all sections centered, the default font and small caps

\usepackage{fancyhdr} % Custom headers and footers
\pagestyle{fancyplain} % Makes all pages in the document conform to the custom headers and footers
\fancyhead{} % No page header - if you want one, create it in the same way as the footers below
\fancyfoot[L]{} % Empty left footer
\fancyfoot[C]{} % Empty center footer
\fancyfoot[R]{\thepage} % Page numbering for right footer
\renewcommand{\headrulewidth}{0pt} % Remove header underlines
\renewcommand{\footrulewidth}{0pt} % Remove footer underlines
\setlength{\headheight}{13.6pt} % Customize the height of the header

\numberwithin{equation}{section} % Number equations within sections (i.e. 1.1, 1.2, 2.1, 2.2 instead of 1, 2, 3, 4)
\numberwithin{figure}{section} % Number figures within sections (i.e. 1.1, 1.2, 2.1, 2.2 instead of 1, 2, 3, 4)
\numberwithin{table}{section} % Number tables within sections (i.e. 1.1, 1.2, 2.1, 2.2 instead of 1, 2, 3, 4)

\setlength\parindent{0pt} % Removes all indentation from paragraphs - comment this line for an assignment with lots of text

\newcommand{\horrule}[1]{\rule{\linewidth}{#1}} % Create horizontal rule command with 1 argument of height





%----------------------------------------------------------------------------------------
%	TÍTULO Y DATOS DEL ALUMNO
%----------------------------------------------------------------------------------------

\title{	
\normalfont \normalsize 
\textsc{\textbf{Asignatura (2016-2017)} \\ Grado en Ingeniería Informática \\ Universidad de Granada} \\ [25pt] % Your university, school and/or department name(s)
\horrule{0.5pt} \\[0.4cm] % Thin top horizontal rule
\huge Título \\ % The assignment title
\horrule{2pt} \\[0.5cm] % Thick bottom horizontal rule
}

\author{Samuel Cardenete Rodríguez} % Nombre y apellidos

\date{\normalsize\today} % Incluye la fecha actual

%----------------------------------------------------------------------------------------
% DOCUMENTO
%----------------------------------------------------------------------------------------

\begin{document}

\maketitle % Muestra el Título

\newpage %inserta un salto de página

\tableofcontents % para generar el índice de contenidos

\listoffigures

\listoftables

\newpage


%----------------------------------------------------------------------------------------
%	CODIGO EJEMPLO
%----------------------------------------------------------------------------------------

\begin{comment}


\begin{lstlisting}[language=bash, style=customc]
warehouses=4
loadWorkers=4
terminals=1
//To run specified transactions per terminal- runMins must equal zero
runTxnsPerTerminal=0
//To run for specified minutes- runTxnsPerTerminal must equal zero
runMins=10
//Number of total transactions per minute
limitTxnsPerMin=300

//Set to true to run in 4.x compatible mode. Set to false to use the
//entire configured database evenly.
terminalWarehouseFixed=true
\end{lstlisting}

\end{comment}


%----------------------------------------------------------------------------------------
%	FIGURA
%----------------------------------------------------------------------------------------

\begin{comment}


\begin{figure}[H]
	\centering
	\includegraphics[scale=0.5]{1.png}  
	\caption{Corridor} \label{fig: Corridor}
\end{figure}


\end{comment}
%----------------------------------------------------------------------------------------
%	TABLA (ancho fijo)
%----------------------------------------------------------------------------------------

\begin{comment}

\begin{table}[htb]
\centering
\begin{tabular}{| p{2.2cm}| p{2.2cm} |  p{2.2cm} |}
\hline
\multicolumn{3}{|c|}{Europa} \\
\hline
\multicolumn{2}{|c|}{Localización} & Estado \\
\hline \hline
España & Madrid & Canada \\ \hline
España & Sevilla & EE.UU \\ \hline
Francia & París & Inglaterra \\ \hline
\end{tabular}
\caption{Tabla de ancho fijo.}
\label{tabla:Título de tabla}
\end{table}

\end{comment}

\bibliography{citas} %archivo citas.bib que contiene las entradas 
\bibliographystyle{plain} % hay varias formas de citar



\end{document}
