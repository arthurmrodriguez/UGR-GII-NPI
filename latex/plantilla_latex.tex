\input{preambuloSimple.tex}




%----------------------------------------------------------------------------------------
%	TÍTULO Y DATOS DEL ALUMNO
%----------------------------------------------------------------------------------------

\title{	
\normalfont \normalsize 
\textsc{\textbf{Asignatura (2016-2017)} \\ Grado en Ingeniería Informática \\ Universidad de Granada} \\ [25pt] % Your university, school and/or department name(s)
\horrule{0.5pt} \\[0.4cm] % Thin top horizontal rule
\huge Título \\ % The assignment title
\horrule{2pt} \\[0.5cm] % Thick bottom horizontal rule
}

\author{Samuel Cardenete Rodríguez} % Nombre y apellidos

\date{\normalsize\today} % Incluye la fecha actual

%----------------------------------------------------------------------------------------
% DOCUMENTO
%----------------------------------------------------------------------------------------

\begin{document}

\maketitle % Muestra el Título

\newpage %inserta un salto de página

\tableofcontents % para generar el índice de contenidos

\listoffigures

\listoftables

\newpage


%----------------------------------------------------------------------------------------
%	CODIGO EJEMPLO
%----------------------------------------------------------------------------------------

\begin{comment}


\begin{lstlisting}[language=bash, style=customc]
warehouses=4
loadWorkers=4
terminals=1
//To run specified transactions per terminal- runMins must equal zero
runTxnsPerTerminal=0
//To run for specified minutes- runTxnsPerTerminal must equal zero
runMins=10
//Number of total transactions per minute
limitTxnsPerMin=300

//Set to true to run in 4.x compatible mode. Set to false to use the
//entire configured database evenly.
terminalWarehouseFixed=true
\end{lstlisting}

\end{comment}


%----------------------------------------------------------------------------------------
%	FIGURA
%----------------------------------------------------------------------------------------

\begin{comment}


\begin{figure}[H]
	\centering
	\includegraphics[scale=0.5]{1.png}  
	\caption{Corridor} \label{fig: Corridor}
\end{figure}


\end{comment}
%----------------------------------------------------------------------------------------
%	TABLA (ancho fijo)
%----------------------------------------------------------------------------------------

\begin{comment}

\begin{table}[htb]
\centering
\begin{tabular}{| p{2.2cm}| p{2.2cm} |  p{2.2cm} |}
\hline
\multicolumn{3}{|c|}{Europa} \\
\hline
\multicolumn{2}{|c|}{Localización} & Estado \\
\hline \hline
España & Madrid & Canada \\ \hline
España & Sevilla & EE.UU \\ \hline
Francia & París & Inglaterra \\ \hline
\end{tabular}
\caption{Tabla de ancho fijo.}
\label{tabla:Título de tabla}
\end{table}

\end{comment}

\bibliography{citas} %archivo citas.bib que contiene las entradas 
\bibliographystyle{plain} % hay varias formas de citar



\end{document}
